\setlength{\LTpost}{0mm}
\begin{longtable}{lcccc}
\caption*{
{\large Tipo de universidad vs Salario}
} \\ 
\toprule
\textbf{Tipo de universidad} & \textbf{1}, N = 185,129 & \textbf{2}, N = 6,401 & \textbf{3}, N = 31,018 & \textbf{4}, N = 6,230 \\ 
\midrule
Sueldo &  &  &  &  \\ 
����1 & 60,488 (33\%) & 1,345 (21\%) & 7,363 (24\%) & 855 (14\%) \\ 
����2 & 43,074 (23\%) & 1,261 (20\%) & 6,086 (20\%) & 1,151 (18\%) \\ 
����3 & 45,965 (25\%) & 1,942 (30\%) & 7,924 (26\%) & 1,949 (31\%) \\ 
����4 & 20,217 (11\%) & 1,087 (17\%) & 5,796 (19\%) & 1,570 (25\%) \\ 
����5 & 4,716 (2.5\%) & 203 (3.2\%) & 1,399 (4.5\%) & 277 (4.4\%) \\ 
����6 & 1,076 (0.6\%) & 116 (1.8\%) & 285 (0.9\%) & 57 (0.9\%) \\ 
����7 & 1,109 (0.6\%) & 113 (1.8\%) & 365 (1.2\%) & 74 (1.2\%) \\ 
����9 & 8,485 (4.6\%) & 334 (5.2\%) & 1,799 (5.8\%) & 297 (4.8\%) \\ 
\bottomrule
\end{longtable}
\begin{minipage}{\linewidth}
Fuente = Encuesta de inserci�n laboral de los titulados universiatrios 2019, INE.\\
\end{minipage}

